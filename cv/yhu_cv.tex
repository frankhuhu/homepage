\documentclass[margin,line]{res}

\usepackage{hyperref}

\oddsidemargin -.5in
\evensidemargin -.5in
\textwidth=6.0in
\itemsep=0in
\parsep=0in
% if using pdflatex:
%\setlength{\pdfpagewidth}{\paperwidth}
%\setlength{\pdfpageheight}{\paperheight} 

\newenvironment{list1}{
  \begin{list}{\ding{113}}{%
      \setlength{\itemsep}{0in}
      \setlength{\parsep}{0in} \setlength{\parskip}{0in}
      \setlength{\topsep}{0in} \setlength{\partopsep}{0in} 
      \setlength{\leftmargin}{0.17in}}}{\end{list}}
\newenvironment{list2}{
  \begin{list}{$\bullet$}{%
      \setlength{\itemsep}{0in}
      \setlength{\parsep}{0in} \setlength{\parskip}{0in}
      \setlength{\topsep}{0in} \setlength{\partopsep}{0in} 
      \setlength{\leftmargin}{0.2in}}}{\end{list}}


\begin{document}

\name{Yongjian Hu \vspace*{.1in}}

\begin{resume}
\section{\sc Contact Information}
\vspace{.05in}
\begin{tabular}{@{}p{2in}p{4in}}
463 Winston Chung Hall              & {\it Mobile:}  (951) 880-8390 \\            
Dept. of Computer Science           & {\it E-mail:}  yhu009@cs.ucr.edu\\       
University of California Riverside  & {\it Homepage:} \url{http://www.cs.ucr.edu/~yhu009}\\
Riverside, CA 92521 
\end{tabular}


\section{\sc Research Interests}
Software reliability, programming languages, static and dynamic analysis, 
debugging, deterministic replay and race detection for event-driven mobile systems, 
static information flow analysis.


\section{\sc Education}
{\bf University of California Riverside}, Riverside, California USA\\
%{\em Department of Statistics} 
\vspace*{-.1in}
\begin{list1}
\item[] Ph.D. Candidate, Computer Science \& Engineering, (2012.9 - Present)
\begin{list2}
\vspace*{.05in}
\item GPA: 3.967/4.0
%\item Dissertation Topic:
\item Advisor:  Prof. Iulian Neamtiu
\end{list2}
\end{list1}

{\bf Donghua University}, Shanghai, China\\
\vspace*{-.1in}
\begin{list1}
\item[] M.S., Computer Science,  2010.3
\item[] B.S., Computer Science,  2007.6
\end{list1}


\section{\sc Experience}
{\bf University of California Riverside}, Riverside, CA USA

\vspace{-.3cm}
{\em Research Assistant} \hfill {\bf 2013.7 - present}\\
Research on applying static and dynamic analysis techniques to detect, verify and debug
event-driven races on mobile devices.
First, I designed a light-weighted record and replay tool called VALERA(OOPSLA'15).
Traditional whole system replay tools record every memory access operation which is
too heavy-weighted and not suitable for mobile devices. VALERA's insight is to capture
only a portion of the critical events then replay them by their original execution order, 
and it's enough for reproducting many event-driven concurrent bugs.

\vspace{-.3cm}
Second, based on VALERA, I develop another tool called EVRA(ISSTA'16) which aim
to reproduce and verify event-driven races. EVRA use EventRacer(OOPSLA'15) as front-end
to collect race report, then use deterministic replay feature provided by VALERA to flip the
events and observe the side effects to filter out false races and benign races.

\vspace{-.3cm}
Third, I'm developing a static analysis tool named SIERRA to statically detect event-driven races.
Unlike dynamic analysis, SIERRA is sound because it tries to reason every possible event order
that may lead to race. SIERRA is based on WALA to generate call graph and pointer analysis,
and it innovatively implements selective hybrid context sensitivity, event dependency graph, static 
happens-before inference and path sensitive refutation techniques to effectively detect races.



{\bf Microsoft Research}, Redmond, WA USA

\vspace{-.3cm}
{\em Research Intern} \hfill {\bf 2016.6 - 2016.9}\\
Leverage program analysis techniques to analyze deep links for mobile applications.
Details cannot be disclosed due to NDA.
Advised by Oriana Riva and Suman Nath.


{\bf Intel Corporation}, Shanghai, China

\vspace{-.3cm}
{\em Software Engineer} \hfill {\bf 2010.4 - 2012.7}\\
Member of Binary Translation Team, SSG. I join the "Houdini" project.
"Houdini" is a dynamic binary translator that translates Arm code to 
x86. It allows Android applications which contain native Arm code to 
seamlessly run on x86 devices. My main job is to develope tools to ensure 
product quality and performance. I have developed a code/path coverage 
tool to provide metrics of the software quality. The performance tool 
is based on sampling and instrumentation techniques to detect and 
breakdown the overhead of the product.


\section{\sc Publications}

Yongjian Hu, Iulian Neamtiu, Arash Alavi. Automatically Verifying and Reproducing Event-based Races in Android Apps.
The International Symposium on Software Testing and Analysis (ISSTA'16), July 2016.

Yongjian Hu, Iulian Neamtiu. VALERA: An Effective and Efficient Record-and-replay Tool for Android.
IEEE/ACM International Conference on Mobile Software Engineering and Systems (MobileSoft 2016), May 2016.

Yongjian Hu, Iulian Neamtiu. Fuzzy and Cross-App Replay for Smartphone Apps.
The 11th IEEE/ACM International Workshop on Automation of Software Test (AST 2016), May 2016.

Yongjian Hu, Tanzirul Azim, Iulian Neamtiu. Versatile yet Lightweight Record-and-Replay for Android.
Proceedings of the 2015 ACM SIGPLAN International Conference on Object-Oriented Programming, Systems, Languages, and Applications(OOPSLA),
Pages 349-366, October 2015.

Yongjian Hu, Tanzirul Azim, Iulian Neamtiu. Improving the Android Development Lifecycle with the VALERA Record-and-replay Approach.
Third International Workshop on Mobile Development Lifecycle(MobileDeli), October 2015.


Yongjian Hu, Wei Xiao. A System-level Path Coverage Tool for Software Validation.
Intel Software Professionals Conference (SWPC), October 2011.

Wei Xiao, Haihao Shen, Yongjian Hu. A Multi-platform System-level Path Coverage Tool.
Intel Design and Test Technology Conference(DTTC), June 2011.


%\section{\sc Papers in preparation}

%\section{\sc Conference Presentations}

\section{\sc Professional Experience}

OOPSLA 2016 Artifact Evaluation Committee.

ICSME 2014 External Reviewer.

QRS 2016 External Reviewer.


\section{\sc Honors and Awards} 
University of California Riverside Graduate Fellowship, 2012.

\vspace*{-2.5mm}
Excellent Graduate Student of Donghua University, 2010.

\vspace*{-2.5mm}
Excellent Undergraduate Student of Shanghai. 2007.

\vspace*{-2.5mm}
Bronze Medal, ACM/ICPC Asia Regional, Beijing 2006.

\vspace*{-2.5mm}
Bronze Medal, ACM/ICPC Asia Regional, Beijing 2005.



\section{\sc Computer Skills} 
\begin{list2}
\item Programming Languages: Java, C/C++, OCaml, Ruby.
\item Tools: Vim, Eclipse, Git, Subversion.
\item Operating Systems: Linux, Mac OSX, Windows.
\end{list2}



\end{resume}
\end{document}




