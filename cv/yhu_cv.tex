\documentclass[margin,line]{res}

\usepackage{hyperref}

\oddsidemargin -.5in
\evensidemargin -.5in
\textwidth=6.0in
\itemsep=0in
\parsep=0in
% if using pdflatex:
%\setlength{\pdfpagewidth}{\paperwidth}
%\setlength{\pdfpageheight}{\paperheight} 

\newenvironment{list1}{
  \begin{list}{\ding{113}}{%
      \setlength{\itemsep}{0in}
      \setlength{\parsep}{0in} \setlength{\parskip}{0in}
      \setlength{\topsep}{0in} \setlength{\partopsep}{0in} 
      \setlength{\leftmargin}{0.17in}}}{\end{list}}
\newenvironment{list2}{
  \begin{list}{$\bullet$}{%
      \setlength{\itemsep}{0in}
      \setlength{\parsep}{0in} \setlength{\parskip}{0in}
      \setlength{\topsep}{0in} \setlength{\partopsep}{0in} 
      \setlength{\leftmargin}{0.2in}}}{\end{list}}


\begin{document}

\name{Yongjian Hu \vspace*{.1in}}

\begin{resume}
\section{\sc Contact Information}
\vspace{.05in}
\begin{tabular}{@{}p{2in}p{4in}}
100 Avenue of the Americas              & {\it Mobile:}  (951) 880-8390 \\
New York, NY 10013                      & {\it E-mail:}  yhu009@cs.ucr.edu\\
                                        & {\it Homepage:} \url{http://www.cs.ucr.edu/~yhu009}\\
\end{tabular}


\section{\sc Research Interests}
Software reliability, programming languages, static and dynamic analysis, 
debugging, deterministic replay and race detection for event-driven mobile systems, 
static information flow analysis.


\section{\sc Education}
{\bf University of California Riverside}, Riverside, California USA\\
%{\em Department of Statistics} 
\vspace*{-.1in}
\begin{list1}
\item[] Ph.D. Computer Science \& Engineering, 2012.9 - 2017.12
\begin{list2}
\vspace*{.05in}
\item Dissertation: Detecting and Verifying Event-Driven Races in Mobile Apps
\item Advisor:  Prof. Iulian Neamtiu
\item GPA: 3.967/4.0
\end{list2}
\end{list1}

{\bf Donghua University}, Shanghai, China\\
\vspace*{-.1in}
\begin{list1}
\item[] M.S., Computer Science,  2010.3
\item[] B.S., Computer Science,  2007.6
\end{list1}


\section{\sc Experience}
{\bf Two Sigma Investments, LP}, New York, NY USA

\vspace{-.3cm}
{\em Software Engineer} \hfill {\bf 2018.1 - present} \\
Software Engineer of model engineering team.

{\bf University of California Riverside}, Riverside, CA USA

\vspace{-.3cm}
{\em Research Assistant} \hfill {\bf 2013.7 - 2017.12}\\
Research on applying static and dynamic analysis techniques to detect, verify and debug
event-driven races on mobile devices.

\vspace{-.3cm}
First, I developed a static analysis tool named SIERRA(ASPLOS'18) to statically detect event-driven races.
There are 3 main research contributions from SIERRA: 1) A new context abstraction called action-sensitivity for
static analysis that works very well for event-driven programs; 2) Static happens-before relation inference
to filter out false races within ordered events; 3) Static backward symbolic execution to filter false races
due to ad-hoc synchronization.

\vspace{-.3cm}
Second, I designed a light-weighted record and replay tool called VALERA(OOPSLA'15).
VALERA's insight is to capture only a portion of the critical
events then replay them by their original execution order, 
and it's enough for reproducting many event-driven concurrent bugs.

\vspace{-.3cm}
Third, based on VALERA, I develop another tool called EVRA(ISSTA'16) which aim
to reproduce and verify event-driven races. The core idea of EVRA is to 
use deterministic replay feature provided by VALERA to flip the
events and observe the side effects to filter out false races and benign races.


{\bf Google}, Sunnyvale, CA USA

\vspace{-.3cm}
{\em Software Engineering Intern} \hfill {\bf 2017.6 - 2017.9}\\
Intern of Google security Laser team. Worked on large scale Android WebView vulnerabilities
detection via static analysis. Advised by Patrick Mutchler and Gogul Balakrishnan.

{\bf Microsoft Research}, Redmond, WA USA

\vspace{-.3cm}
{\em Research Intern} \hfill {\bf 2016.6 - 2016.9}\\
Worked on automatically extract deep links for mobile apps via path-selective taint analysis.
Designed a fully static analysis tool that leveraged state-of-art taint analysis and symbolic execution.
Advised by Oriana Riva and Suman Nath.

{\bf Intel Corporation}, Shanghai, China

\vspace{-.3cm}
{\em Software Engineer} \hfill {\bf 2010.4 - 2012.7}\\
Member of Binary Translation Team, worked in the "Houdini" project.
"Houdini" is a dynamic binary translator that translates Arm code to 
x86. It allows Android applications which contain native Arm code to 
seamlessly run on x86 devices. My main job is to develope tools to ensure 
product quality and performance. I have developed a code/path coverage 
tool to provide metrics of the software quality. The performance tool 
is based on sampling and instrumentation techniques to detect and 
breakdown the overhead of the product.


\section{\sc Publications}
Yongjian Hu, Iulian Neamtiu.
Static Detection of Event-based Races in Android Apps.
The 23rd ACM International Conference on Architectural Support for Programming Languages and Operating Systems (ASPLOS'18), March 2018.

Yunfeng Hong, Yongjian Hu, Chun-Ming Lai, Felix Wu, Iulian Neamtiu, Yu Paul, Hasan Cam, Gail-Joon Ahn.
Defining and Detecting Environment Discrimination in Android Apps.
The 13th EAI International Conference on Security and Privacy in Communication Networks (SecureComm'17), October 2017.

Iulian Neamtiu, Xuetao Wei, Michalis Faloutsos, Lorenzo Gomez, Tanzirul Azim, Yongjian Hu, Zhiyong Shan.
Improving Smartphone Security and Reliability.
Journal of Interconnection Networks, Volume 17, Issue 01, March 2017

Yongjian Hu, Iulian Neamtiu, Arash Alavi. Automatically Verifying and Reproducing Event-based Races in Android Apps.
The International Symposium on Software Testing and Analysis (ISSTA'16), July 2016.

Yongjian Hu, Iulian Neamtiu. VALERA: An Effective and Efficient Record-and-replay Tool for Android.
IEEE/ACM International Conference on Mobile Software Engineering and Systems (MobileSoft 2016), May 2016.

Yongjian Hu, Iulian Neamtiu. Fuzzy and Cross-App Replay for Smartphone Apps.
The 11th IEEE/ACM International Workshop on Automation of Software Test (AST 2016), May 2016.

Yongjian Hu, Tanzirul Azim, Iulian Neamtiu. Versatile yet Lightweight Record-and-Replay for Android.
Proceedings of the 2015 ACM SIGPLAN International Conference on Object-Oriented Programming, Systems, Languages, and Applications(OOPSLA),
Pages 349-366, October 2015.

Yongjian Hu, Tanzirul Azim, Iulian Neamtiu. Improving the Android Development Lifecycle with the VALERA Record-and-replay Approach.
Third International Workshop on Mobile Development Lifecycle(MobileDeli), October 2015.


Yongjian Hu, Wei Xiao. A System-level Path Coverage Tool for Software Validation.
Intel Software Professionals Conference (SWPC), October 2011.

Wei Xiao, Haihao Shen, Yongjian Hu. A Multi-platform System-level Path Coverage Tool.
Intel Design and Test Technology Conference(DTTC), June 2011.


%\section{\sc Papers in preparation}

%\section{\sc Conference Presentations}


%\section{\sc Patents}
%Microsoft Patent 410766-US-NP: Micro Service Framework Derived From Third Party Apps.
%Inventors: Yongjian Hu, Oriana Riva, Suman Nash, and Doug Burger.

\section{\sc Professional Experience}
\begin{list2}
\item PLDI 2018 PLDI Research Artifacts Committee.
\item Software Quality Journal Reviewer.
\item OOPSLA 2017 Artifact Evaluation Committee.
\item OOPSLA 2016 Artifact Evaluation Committee.
\item ICSME 2014 External Reviewer.
\item QRS 2016 External Reviewer.
\end{list2}

\section{\sc Honors and Awards} 
\begin{list2}
\item Dissertation Year Program Fellowship, UC Riverside, 2017.
\item University of California Riverside Graduate Fellowship, 2012.
\item Excellent Graduate Student of Donghua University, 2010.
\item Excellent Undergraduate Student of Shanghai. 2007.
\item Bronze Medal, ACM/ICPC Asia Regional, Beijing 2006.
\item Bronze Medal, ACM/ICPC Asia Regional, Beijing 2005.
\end{list2}


\section{\sc Computer Skills} 
\begin{list2}
\item Programming Languages: Java, C/C++, OCaml, Ruby.
\item Tools: Vim, Eclipse, Git, Subversion.
\item Operating Systems: Linux, Mac OSX, Windows.
\end{list2}



\end{resume}
\end{document}




